% !TEX encoding   = UTF8
% !TEX spellcheck = en_US

%%% Tables
% для вертикального центрирования ячеек в tabulary
\def\zz{\ifx\[$\else\aftergroup\zzz\fi}
%$ \] % <-- чиним подсветку синтаксиса в некоторых редакторах
\def\zzz{\setbox0\lastbox
  \dimen0\dimexpr\extrarowheight + \ht0-\dp0\relax
  \setbox0\hbox{\raise-.5\dimen0\box0}%
  \ht0=\dimexpr\ht0+\extrarowheight\relax
  \dp0=\dimexpr\dp0+\extrarowheight\relax
  \box0
}


%%%  Чересстрочное форматирование таблиц
%% http://tex.stackexchange.com/questions/278362/apply-italic-formatting-to-every-other-row
\newcounter{rowcnt}
\newcommand\altshape{\ifnumodd{\value{rowcnt}}{\color{red}}{\vspace*{-1ex}\itshape}}
% \AtBeginEnvironment{tabular}{\setcounter{rowcnt}{1}}
% \AtEndEnvironment{tabular}{\setcounter{rowcnt}{0}}


%%% Listings
\usemintedstyle{manni}

\renewcommand{\theFancyVerbLine}{%
  \textcolor{gray}{\tiny\arabic{FancyVerbLine}}%
}

\newmintinline{cpp}{fontsize=\small}
\newmintinline{text}{fontsize=\small}

\newmint{cpp}{fontsize=\small}
\newmint{js}{fontsize=\small}
\newmint[txt]{text}{fontsize=\small}
\newmint{console}{fontsize=\small}

\newminted{cpp}{linenos, fontsize=\small, numbersep=0.2em}
\newminted{text}{fontsize=\small, tabsize=2}
\newminted{console}{fontsize=\small, tabsize=2}

\newmintedfile{cpp}{linenos, fontsize=\small, numbersep=0.2em}
\newmintedfile{js}{linenos, fontsize=\small, numbersep=0.2em}
\newmintedfile{text}{fontsize=\small, tabsize=2}
\newmintedfile{console}{fontsize=\small, tabsize=2}


%%% Русская традиция начертания математических знаков
\renewcommand{\le}{\ensuremath{\leqslant}}
\renewcommand{\leq}{\ensuremath{\leqslant}}
\renewcommand{\ge}{\ensuremath{\geqslant}}
\renewcommand{\geq}{\ensuremath{\geqslant}}
\renewcommand{\emptyset}{\varnothing}

%%% Русская традиция начертания математических функций (на случай копирования из зарубежных источников)
\renewcommand{\tan}{\operatorname{tg}}
\renewcommand{\cot}{\operatorname{ctg}}
\renewcommand{\csc}{\operatorname{cosec}}

%%% Русская традиция начертания греческих букв (греческие буквы вертикальные, через пакет upgreek)
\renewcommand{\epsilon}{\ensuremath{\upvarepsilon}}   %  русская традиция записи
\renewcommand{\phi}{\ensuremath{\upvarphi}}
%\renewcommand{\kappa}{\ensuremath{\varkappa}}
\renewcommand{\alpha}{\upalpha}
\renewcommand{\beta}{\upbeta}
\renewcommand{\gamma}{\upgamma}
\renewcommand{\delta}{\updelta}
\renewcommand{\varepsilon}{\upvarepsilon}
\renewcommand{\zeta}{\upzeta}
\renewcommand{\eta}{\upeta}
\renewcommand{\theta}{\uptheta}
\renewcommand{\vartheta}{\upvartheta}
\renewcommand{\iota}{\upiota}
\renewcommand{\kappa}{\upkappa}
\renewcommand{\lambda}{\uplambda}
\renewcommand{\mu}{\upmu}
\renewcommand{\nu}{\upnu}
\renewcommand{\xi}{\upxi}
\renewcommand{\pi}{\uppi}
\renewcommand{\varpi}{\upvarpi}
\renewcommand{\rho}{\uprho}
%\renewcommand{\varrho}{\upvarrho}
\renewcommand{\sigma}{\upsigma}
%\renewcommand{\varsigma}{\upvarsigma}
\renewcommand{\tau}{\uptau}
\renewcommand{\upsilon}{\upupsilon}
\renewcommand{\varphi}{\upvarphi}
\renewcommand{\chi}{\upchi}
\renewcommand{\psi}{\uppsi}
\renewcommand{\omega}{\upomega}

\def\slantfrac#1#2{ \hspace{3pt}\!^{#1}\!\!\hspace{1pt}/
  \hspace{2pt}\!\!_{#2}\!\hspace{3pt}
} %Макрос для красивых дробей в строчку (например, 1/2)

\DeclareMathOperator{\Div}{div}      % divergence of a vector
\DeclareMathOperator{\rot}{rot}      % rotation of a vector
\DeclareMathOperator{\diag}{diag}    % diagonal matrix
\DeclareMathOperator{\col}{col}      % columns
\DeclareMathOperator{\const}{const}  % constant
\newcommand*{\pfrac}[2]{\ensuremath{\frac{\partial #1}{\partial #2}}}

\newcommand*{\tcol}[1]{\begin{tabular}{@{}c@{}}#1\end{tabular}}
\newcommand*{\Chapter}[1]{\chapter*{#1}\addcontentsline{toc}{chapter}{#1}}